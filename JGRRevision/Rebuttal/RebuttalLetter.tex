\documentclass[10pt,a4paper]{letter}
\usepackage[utf8]{inputenc}
\usepackage{amsmath}
\usepackage{amsfonts}
\usepackage{amssymb}
\usepackage[left=2cm,right=2cm,top=2cm,bottom=2cm]{geometry}

\begin{document}

\signature{Trever T. Hines}

\begin{letter}{}
\opening{Dear Editor,}

We were pleased with the positive comments from you and the reviewers, and we feel that the comments brought up valid concerns. We hope that we have addressed these concerns sufficiently in our revision. In order to fully address the comments, we have included a supplementary material document.  We have also made some changes to the manuscript which were not specifically requested by the reviewers.  Most of these changes were stylistic and do not change the content of the manuscript.  One of the more substantial changes was that we merged Figures 7 and 14, which are both showing observed and predicted radial components of postseismic displacements, facilitating direct comparison of the figures. 

We added supporting figures S3 and S4, which break down the predicted displacements of our preferred model into an elastic and viscoelastic component.  We feel that these figures help support our discussion on lines 509 to 523, where we attribute inferences of fault slip and viscoelastic relaxation to different aspects of the observed postseismic displacements.  

On line 500 to 502 of our previous version of this manuscript we make the claim that “After one year, afterslip is inferred to be deeper down on the Sierra Cucapah segment, which is describing much of the sustained near-field postseismic deformation.”  This sentence is not accurate of the preferred model we present in the manuscript. As can be clearly seen in the new figures S3-S4, afterslip is describing some of the later near-field deformation, although most of it is being described by viscoelastic relaxation.  We have reworded this passage and updated the introduction and conclusion to accurately describe our interpretations of the mechanisms driving later near-field deformation.  This does not significantly alter the main conclusions of our manuscript, because we are very clear, both in the original and revised versions, that the deformation mechanism driving later near-field deformation cannot be definitively resolved with these data.        

Each of the reviewer comments are included below followed by our response and a description of how these comments were addressed in this revised manuscript.

\textbf{Reviewer \#1 Evaluations:}\newline
\textbf{Significant: The paper has some unclear or incomplete reasoning but will likely be a significant contribution with revision and clarification.}\newline
\textbf{Supported: Yes}\newline
\textbf{Referencing: Yes}\newline
\textbf{Quality: Yes, it is well-written, logically organized, and the figures and tables are appropriate.}\newline
\textbf{Data: Yes}

\textbf{Reviewer \#1 (Formal Review for Authors (shown to authors)):}\newline
\textbf{In this manuscript, the authors carry out an in-depth analysis of deformation time-series following the 2010 El Mayor-Cucapah earthquake, modeling simultaneously coseismic effects, afterslip and viscoelastic relaxation. Their modeling approach is technically sound and well discussed. In particular, I appreciated that the authors clearly explain which steps of their approach are subjective decisions.}

\textbf{I think that this work is of potential interest for a broad range of researchers and therefore I recommend its publication on JGR. I have just a few minor comments, which I believe that will further improve the manuscript.}

\textbf{- The only aspect of the modeling that is not discussed is the resolution of the inversions. For both coseismic slip and afterslip, faults are discretized with a 4km step, but the real resolving power of the dataset is likely to be lower. I suggest showing a checkerboard test, maybe as additional material to avoid introducing too much clutter in the main text. A brief discussion the resolution on viscosity values would also be helpful.}

We have included the results of a synthetic test as Figure S2 in the supplementary information which illustrates the resolvability of fault slip and lithospheric viscosity for the inversion in Section 3.2.  In this synthetic test, we generate displacements resulting from a checkerboard pattern of coseismic slip and viscoelastic relaxation in an elastic lithosphere overlying a uniformly low viscosity asthenosphere.  Our checkerboard test illustrates the effect of regularization on our solution and we believe that it should provide the reader with a better sense of the uncertainty in our solution.  In lines 416-418 and 437 to 440, we comment on the synthetic test and use it to explain how the reader should interpret our inferred solution in Section 3.2. 

\textbf{- I think that it would be useful to show the regularization tradeoff curves for both elastic and full-postseismic inversion, in order to quantify the effect of the penalty parameters. These curves could be shown as additional material.}

We include the trade-off curves as Figure S1 in the supplementary information and we explain how we used the tradeoff curves to infer the two penalty parameters from Section 3.2.

\textbf{- The introduction of a Zener rheology undoubtedly improves the agreement with data, but in many stations the model still fails to explain the observed signal. It would be useful to give a statistical measure of the significance of misfit reduction in relation to the additional model parameters, to ensure that the lower chi-square is meaningful and not just an over-fitting.} 

We do not know of a statistical measure that would be appropriate for demonstrating the statistical significance of using a Zener rheology instead of a Maxwell or elastic rheology.  The Zener model includes one extra model parameter compared to the Maxwell model, while  the total number of model parameters is on the order of a few hundred, resulting in very small, arguably trivial, change to the degrees of freedom. We do not feel that an F-test would be appropriate because the uncertainty on our data is likely underestimated, which we discuss in Section 2, and combined with the fact that only one model parameter was added, this results in an F-tests which we feel is overly confident.  We performed the F-test to see how significant the Zener model is over the Maxwell model, and the F-test suggests that we can reject the null hypothesis with effectively 100\% confidence. We think that comparing the models with an F-test can be misleading and we instead think that a visual comparison between the observed and predicted displacements, which are presented in Figures 4, 5, and 7, is a more honest means of appraisal.  Alternatively, we could compare the Zener model to the Maxwell or elastic model with the Akaike Information Criterion (AIC).  The AIC is a function of the number of model parameters and the chi-squared misfit.  The number of model parameters remains effectively constant for the Zener and Maxwell model, and thus comparing the AIC of the models does not offer any more information than comparing their chi-squared values, which we present in the text.

\textbf{Minor points:}

\textbf{- I suggest adding a paragraph in the introduction to describe the earthquake and its seismo-tectonic context.}

We do not describe the seismo-tectonic context of the El Mayor-Cucapah earthquake because we do not feel that such general background information is necessary to understand our paper. Likewise, our postseismic results do not yield deeper understanding of the seismo-tectonics of the area. Others have already described the El Mayor in context of the seismo-tectonics of the region, and on line 39 we now point the reader to two papers doing so.

\textbf{- Magnitudes would read better if the notation "Mw=7.2" or "Mw$>$7" is used instead of "Mw7.2" or "$>$Mw7".}

We have made the suggested changes.

\textbf{- In eqs. (1) and (2), it is not explicitly stated that t is expressed in years.}

We added a sentence stating this on line 116.

\textbf{- line 202: I suggest to quantify the "elevated rate"}

The text has been changed so that we quantify this elevated rate as “a few millimeters per year” on line 204.

\textbf{- Figures 4 ad 5: add a symbol to mark the epicenter}

We have added focal mechanisms at the El Mayor-Cucapah epicenter for Figure 4.  We do not change Figure 5 because the epicenter is not in the frame.

\textbf{- lines 259-260: "Table 3" should read "Table 1"}

We have made the suggested changes.

\textbf{- Figures 6, 12 and 15: a label with the two fault names would be useful. White arrows for the slip direction are difficult to see, maybe it is better to use a more contrasting color (I would try red)}

We have increased the size of the slip vectors on these figures. We also added the fault segment names to these figures.

\textbf{- line 513: `the' should be `The'}

We have made the suggested changes.

\textbf{Reviewer \#2 Evaluations:}\newline
\textbf{Significant: Yes, the paper is a significant contribution and worthy of prompt publication.}\newline
\textbf{Supported: Yes}\newline
\textbf{Referencing: Mostly yes, but some additions are necessary.}\newline
\textbf{Quality: Yes, it is well-written, logically organized, and the figures and tables are appropriate.}\newline
\textbf{Data: Yes}

\textbf{Reviewer \#2 (Formal Review for Authors (shown to authors)):}

\textbf{This is a timely study of the mechanisms of postseismic relaxation following the 2010 M7.2 El Mayor-Cucapah earthquake. The authors make use of ~5 years of postseismic time series and model the time-dependent crustal motions with joint afterslip and lower crust/upper mantle viscoelastic relaxation. This approach is well motivated by observations of both processes following past earthquakes in both megathrust and continental strike-slip settings. The authors leverage the power of several years of observations and perform a careful search of model parameter space to infer rheological parameters. They make the case for non-Maxwellian rheology in the upper mantle. The resulting estimate of Zener-rheology mantle viscosity is consistent with the transient or Maxwell viscosity obtained from other western US postseismic relaxation studies and lacustrine lake unloading studies.}

\textbf{Specific comments}

\textbf{Figures 4 and 5. I cannot figure out what the predicted vertical displacements are. "... shown within the circles at the base of each vector." - are these the black circles? Should the reader be looking at the size of these circles? How do I tell the sign of the predicted vertical displacement? At many stations I don't even see these circles! Also, do the observed horizontal displacements include error ellipses?}

The Figure 4 caption in the previous version of the text was ambiguous and we changed it to read “The black error ellipses show the 68\% confidence interval for the observed horizontal displacements.  Observed vertical displacements are shown as an interpolated field and predicted vertical displacements are shown within the green circles.”  We believe this addresses the reviewers confusion. 

\textbf{Figure 11. The confidence intervals for viscosity seem remarkably tight given the likely small sensitivity to viscosity at, say, the 150-180 km depth interval. Please elaborate on how these uncertainties were determined.}

We appreciate that the reviewer brought up this point and we feel that it is important to emphasize in the text that the uncertainty is likely underestimated due to our added regularization constraints.  We modified the text to further stress this point in lines 433 to 440.  The uncertainties were calculated using the standard bootstrapping technique, which we state.    

\textbf{Section 3.3. After about line 479, remind the reader where the parameters of the preferred rheology model are summarized (Table 1).}

We changed the text as suggested on line 493.

\textbf{Last paragraph of section 3.3. Is there anything in the fits of Figures 4 and 5 that would suggest any systematic pattern in the residuals? I have in mind possible differences between the Salton trough and surrounding regions, which may be expected from the relatively high heat flow in the Salton trough.}

We do not observe any systematic residuals in the Salton Trough that would suggest lateral heterogeneity in the material properties.  We modified the text to state this in lines 525 to 530.

\textbf{Last paragraph of Discussion. Please note that Pollitz (2003) has been superseded by Pollitz (2015), which models post-Hector Mine motions with joint afterslip and viscoelastic relaxation. Pollitz (2015) interprets the early post-Hector Mine motions as predominantly afterslip, and the estimated transient and Maxwell viscosities are ~10 times higher than those of Pollitz (2003).}

We thank the reviewer for pointing out this paper, which nicely complements our work.  We modified the discussion to note the contribution by Pollitz (2015) on line 624 to 627.   

\textbf{Reference}

\textbf{Pollitz, F.F., Post-earthquake relaxation evidence for laterally variable viscoelastic structure and water content in the southern California mantle, J. Geophys. Res., 120, 2672-2696, (2015).}

\closing{thank you for your consideration,}
\end{letter}
\end{document}