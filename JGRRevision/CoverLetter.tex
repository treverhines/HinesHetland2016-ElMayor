\documentclass[12pt,letterpaper]{letter}

\oddsidemargin -0.3in
\evensidemargin -0.3in
\textwidth 7.1in
\topmargin -0.5in
\textheight 9in

\begin{document}

\address{Department of Earth and Environmental Sciences\\
  University of Michigan\\
  1100 North University Ave.\\
  Ann Arbor, MI 48109\\}

\signature{Trever T. Hines}

\begin{letter}{Journal of Geophysical Research: Solid Earth}
\opening{Dear Editor,}

We are submitting this manuscript, ``Rheologic constraints on the upper mantle from five years of postseismic deformation following the El Mayor-Cucapah earthquake'', for consideration for publication in
Journal of Geophysical Research: Solid Earth.  The first author is a PhD student in his fourth year and this manuscript will be the third chapter of his dissertation.  In this manuscript, we analyze displacement time series for GPS stations within 400 km of the 2010 El Mayor-Cucapah earthquake. After estimating postseismic displacements at each station, we observe transient deformation over a much broader region than has been recognized in previous studies on this earthquake.  We attempt to describe the mechanisms causing this deformation using our recently published technique.  We find that a transient viscoelastic rheology, rather than a fluid-like rheology (e.g. Maxwell viscoelasticity), is needed to describe the far-field postseismic deformation.  We are particularly excited by this result because our preferred model does not conflict with the steady-state viscosities inferred from geophysical processes occurring over longer time scales, which is a problem with most postseismic studies.             


\closing{Thank you for your consideration,}

\end{letter}
\end{document}
