\documentclass[12pt]{article}

\usepackage{amssymb,amsmath}
\usepackage[margin=1.0in]{geometry}
\usepackage{fancyhdr} % required for custom header
\usepackage{graphicx}
\usepackage{listings}
\usepackage{courier}
\usepackage[usenames,dvipsnames]{color}

%set up the header
\pagestyle{fancy}
\lhead{Trever Hines}
\chead{El Mayor Postseismic}
\rhead{\today}

\setlength{\headheight}{15pt}
\renewcommand\headrulewidth{1.0pt} % Size of the header rule

%% Title
%%------------------------------------------------------------------------------
\title{	
El Mayor Postseismic
\author{Trever Hines}
\rule{\headwidth}{1.0pt}
}


\begin{document}
\maketitle
\section*{Introduction}

\section*{Data Processing}

  We use continuous GPS position time series provided by University
  Navstar Consortium (UNAVCO) for Plate Boundary Observatory (PBO)
  stations within a 400 km radius about the El Mayor-Cucapah
  epicenter. Our analysis is on the coseismic and postseismic
  deformation resulting from the EMC earthquake, which we collectively
  describe as $u(t)$. We consider GPS position time series,
  $u_\mathrm{obs}(t)$, to be the superposition of $u(t)$, secular
  tectonic deformation, annual and semi-annual fluctuations, and
  coseismic offsets from sigificant earthquakes over the time span of
  this study.  The June 14, 2010 Mw5.8 Ocotillo earthquake and the
  August 26, 2012 Brawley swarm, which consisted of a Mw5.5 and Mw5.3
  event, are the only earthquakes after the EMC earthquake that
  produced noticeable offsets recorded by GPS. Although the Ocotillo
  earthquake had its own series of aftershocks (Haukson), neither
  earthquake produced transient deformation that is detectable with
  GPS. We thus model $u_\mathrm{obs}(t)$ as 
  \begin{equation}\label{TimeSeriesModel}
  \begin{split}  
    u_\mathrm{obs}(t) = &u(t) + c_0 + c_1t + \\
                       &c_2\sin(2\pi t) + c_3\cos(2\pi t) + c_4\sin(4\pi t) + c_5\cos(4\pi t) + \\
                       &c_6H(t-t_\mathrm{oc}) + c_7H(t-t_\mathrm{bs}) + \epsilon.
  \end{split}
  \end{equation}
  Where $t_\mathrm{oc}$ and $t_\mathrm{bs}$ are the times of the
  Ocotillo earthquake and Brawley swarm respectively, $H(t)$ is the
  Heaviside function, $c_0$ through $c_7$ are unknown coefficients,
  and $\epsilon$ is noise with zero mean and variance that is assumed
  known.

  Stations which recorded signals that clearly cannot be described by
  the aforementioned processes are not included in our analysis. This
  includes stations in the Los Angeles basin, which record deformation
  that is largely anthropogenic. In order to ensure an accurate
  estimation of the secular deformation, we only use stations that
  were installed at least six months prior to El Mayor-Cucapah
  earthquake. While several stations were installed after the EMC
  earthquake to improve the spatial resolution of postseismic
  deformation \cite{S2015}, our inverse method uses postseismic
  displacements rather than velocities (e.g. Pollitz), which requires
  the knowledge of the stations preseismic position. Despite our
  inability to utilize potentially rich data, we prefer to use
  postseismic displacements rather than potentially dubious estimates
  of postseismic velocities.

  The October 16, 1999 Hector Mine earthquake, which occurred within
  our study region about 270 km north of the EMC epicenter, has
  produce transient postseismic deformation which we do not wish to
  model either mechanically or through empirical line fitting. We thus
  restrict our analysis to deformation observed six years after the
  Hector Mine earthquake, past which point postseismic deformation for
  nearfield sites occurs at an approximately steady rate
  \cite{SS2009}. When considering stations further away from the
  Hector Mine epicenter, postseismic transience persists for only
  about two years \cite{S2015}.

  We do not assume a parametric form for $u(t)$, (e.g. \cite{R2015}),
  but rather we model $u(t)$ as integrated Brownian motion, so that
  \begin{equation}
    \dot{u}(t) = \sigma^2\int_0^t w(s) ds.
  \end{equation}    
  where $w(t)$ is white noise and the variance of $\dot{u}(t)$
  increases linearly with time by a factor of $\sigma^2$. We use a
  Kalman filtering approach to estimate $u(t)$ and the unknown
  parameters in eq. \ref{TimeSeriesModel} which we describe now. 

  In the context of Kalman filtering, our time varying state vector is
  \begin{equation}
    \mathbf{X}(t) = [u(t),\dot u(t), c_0, ..., c_7]
  \end{equation}
  and eq. \ref{TimeSeriesModel} is the observation function which maps
  the state vector to the GPS observations. We initiate the Kalman
  filter by assuming a prior estimate of $\mathbf{X}(t)$ at time $t_0$ which
  has a sufficiently large covariance to effectively make our prior
  uninformed.  For each time epoch, $t_i$, Bayesian linear regression
  is used to incorperate GPS derived estimates of displacement with
  our prior estimate of the state, $\mathbf{X}_{i|i-1}$, to form a postserior
  estimate of the state, $\mathbf{X}_{i|i}$, which has covariance
  $\mathbf{\Sigma}_{i|i}$.  

  We then use the posterior estimate of the state at time $t_i$ to
  form a prior estimate of the state at time $t_{i+1}$ through the
  transition function
  \begin{equation}\label{predict}
    \mathbf{X}_{i+1|i} = \mathbf{F}_{i+1}\mathbf{X}_{i|i} + \mathbf{\delta}_{i+1} 
  \end{equation}
  where 
  \begin{equation}
    \mathbf{F}_{i+1} = 
    \left[
    \begin{array}{ccc}
    1           & (t_{i+1} - t_i) & \mathbf{0}\\
    0           & 1              & \mathbf{0}\\
    \mathbf{0}  & \mathbf{0}     & \mathbf{I}
    \end{array}
    \right]
  \end{equation}
  and $\mathbf{\delta}_{i+1}$ is the process noise, which has zero
  mean and covariance described by
  \begin{equation}
    \mathbf{Q}_{i+1} = 
    \sigma^2 \left[
    \begin{array}{ccc}
    \frac{(t_{i+1} - t_i)^3}{3} & \frac{(t_{i+1} - t_{i})^2}{2} & \mathbf{0}\\
    \frac{(t_{i+1} - t_i)^2}{2} & (t_{i+1} - t_{i}) & \mathbf{0}\\ 
    \mathbf{0} & \mathbf{0} & \mathbf{0}
    \end{array}
    \right].
  \end{equation}

  The covariance of the new prior state, $\mathbf{X}_{i+1|i}$, is then described by
  \begin{equation}
    \mathbf{\Sigma}_{i+1|i} = \mathbf{F}_{i+1}\mathbf{\Sigma}_{i|i}\mathbf{F}^T_{i+1} + \mathbf{Q}_{i+1}.
  \end{equation}

  This process is repeated for each of the $N$ time epochs at which
  point we use Rauch-Tung-Striebel smoothing to find $X_{i|N}$, which
  is an estimate of the state at time $t_i$ that incorporates all $N$
  GPS observation.  Our final estimates of $u(t)$ are used in
  subsequent analysis, while the remaining components of of the state
  vector are considered nuisance parameters. In the interests of
  computational tractability, we downsample our smoothed time series
  from daily solutions down to weekly solutions.

  We illustrate the effect of filtering in figure ().  We assume a
  constant $\sigma^2$ for each station, which is chosen to be just
  large enough for the earliest transient deformation at the most
  nearfield site, P496?, to be faithfully described by the filtered
  solution.

  Cite Freed 2007 for far reaching postseismic after Landers/Hector Mine.

\section*{Observations}

\section*{Prior Studies}

\section*{Postseismic Modeling}
  We use a fault geometry from \cite{W2011}, which was determined
  using teleseismic, GPS, and InSAR data.  

  After the El Mayor-Cucapah earthquake, additional GPS stations were
  installed In Baja California to record postseismic deformation with
  better spatial coverage.  Two of the stations PTAX and PHJX were
  installed near the epicenter in the Cucapah Mountains.  We do not
  include these two stations in our analysis because the geometry of
  the faults that ruptured during the earthquake is more complicated
  than our assumed fault geometry \cite{O2012} and the near field
  stations would be most sensitive to error in our geometry. We also
  leave of these near field stations in order to avoid any near field
  processes which we do not consider in this paper \cite{GO2014}.

   
  Numerous stations exhibit extraneous signals which can be at  
  \cite{A2007}

\section*{Rheological constraints}

\section*{Results}
\section*{Acknowledgements}
  This material is based on EarthScope Plate Boundary Observatory data
  services provided by UNAVCO through the GAGE Facility with support
  from the National Science Foundation (NSF) and National Aeronautics
  and Space Administration (NASA) under NSF Cooperative Agreement
  No. EAR-1261833.

\begin{thebibliography}{}

\bibitem[Freed et al. (2007)]{F2007} 1. Freed AM, Bürgmann R, Herring
  T. Far-reaching transient motions after Mojave earthquakes require
  broad mantle flow beneath a strong crust. Geophys Res
  Lett. 2007;34:1-5. doi:10.1029/2007GL030959.

\bibitem[Rollins et al. (2015)]{R2015} Rollins, C., Barbot, S.,
  Avouac, J.P. Postseismic Deformation Following the 2010 M7.2 El
  Mayor-Cucapah Earthquake: Observations, Kinematic Inversions, and
  Dynamic Models. 2015. \textit{Pure Appl Geophys.}
  doi:10.1007/s00024-014-1005-6.

\bibitem[Savage and Svarc (2009)]{SS2009} Savage, J.C. and Svarc,
  J.L. Postseismic relaxation following the 1992 M 7 . 3 Landers and
  1999 M 7.1 Hector Mine earthquakes, southern California. 2009. 114,
  1-12. doi:10.1029/2008JB005938.

\bibitem[Spliner et al. (2015)]{S2015} Spliner, J.C., Bennett, R.A.,
  Walls, C., Lawrence, S., Garcia, J.J.G. Assessing long-term
  postseismic deformation following the M7.2 April 2010, El
  Mayor-Cucapah earthquake with implications for lithospheric rheology
  in the Salton Trough. 2015. \textit{J Geophys Res.}
  2015. doi:10.1002/2014JB011613.Received.

\bibitem[Gonzalez-Ortega et al. (2014)]{GO2014} Gonzalez-Ortega A.,
  Fialko, Y., Sandwell D., et al. El Mayor-Cucapah (mw7.2) earthquake:
  Early near-field postseismic deformation from the InSAR and GPS
  observations. 2014. \textit{J Geophys Res.} 119,
  1482-1497. doi:10.1002/2013JB010193

\bibitem[Oskin et al. (2012)]{O2012} Oskin, M.E., Arrowsmith, R.,
  Corona A.H., et al. 2012. Near-Field Deformation from the El
  Mayor-Cucapah Earthquake Revealed by Differential
  LIDAR. \textit{Science}. February,
  702-705. doi:10.1126/science.1213778.

\bibitem[Wei et al. (2011)]{W2011} Wei, S., Fielding, E., Leprince,
  S., et al 2011. Superficial simplicity of the 2010 El Mayor–Cucapah
  earthquake of Baja California in Mexico. \textit{Nat
  Geosci}, 4, 615-618, doi: 10.1038/ngeo1213.

\bibitem[Aagard et al. (2013)]{A2007} Aagaard, B.T., Knepley, M.G. \&
  Williams, C.A., 2013. A domain decomposition approach to
  implementing fault slip in finite-element models of quasi-static and
  dynamic crustal deformation, \textit{J. Geophys.  Res. Solid Earth},
  118, doi: 10.1002/jgrb.50217.

\end{thebibliography}
\end{document}

